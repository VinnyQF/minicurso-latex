\documentclass{article}

\usepackage{geometry}
\usepackage{microtype}
\usepackage{graphicx}
\usepackage[brazil]{babel}
\usepackage[T1]{fontenc}
\usepackage{lmodern}
\usepackage{listings}
\usepackage{setspace}
\usepackage[colorlinks=true, linkcolor=blue]{hyperref}
\usepackage{fancyhdr}
\setlength{\parindent}{0pt}
\pagestyle{fancy} 

% Limpa os campos padrão
\fancyhf{}



% Rodapé
\fancyfoot[L]{Centro Universitário Jorge Amado}
\fancyfoot[C]{\thepage}
\fancyfoot[R]{Vinícius Queiroz Fonseca}

\renewcommand{\headrulewidth}{0.4pt}
\renewcommand{\footrulewidth}{0.4pt}  



\lstset{
	basicstyle=\ttfamily\small,
	frame=single,
	breaklines=true,
	showstringspaces=false,
	extendedchars=true,
	inputencoding=utf8,
	literate=
	{á}{{\'a}}1
	{à}{{\`a}}1
	{ã}{{\~a}}1
	{â}{{\^a}}1
	{é}{{\'e}}1
	{ê}{{\^e}}1
	{í}{{\'i}}1
	{ó}{{\'o}}1
	{ô}{{\^o}}1
	{õ}{{\~o}}1
	{ú}{{\'u}}1
	{ç}{{\c c}}1
	{Á}{{\'A}}1
	{À}{{\`A}}1
	{Ã}{{\~A}}1
	{Â}{{\^A}}1
	{É}{{\'E}}1
	{Ê}{{\^E}}1
	{Í}{{\'I}}1
	{Ó}{{\'O}}1
	{Ô}{{\^O}}1
	{Õ}{{\~O}}1
	{Ú}{{\'U}}1
	{Ç}{{\c C}}1
}

\geometry{
	a4paper,
	left=3cm,
	right=2cm,
	top=3cm,
	bottom=2cm
}


% Configurações da Capa
\author{Vinícius Queiroz Fonseca}
\date{14 de Janeiro de 2026}
\title{Aula 1 - Atividade \LaTeX \hspace{1pt}}

\makeatletter
\renewcommand{\maketitle}{
	\begin{titlepage}
		\centering
		
		\includegraphics[width=5cm]{IMAGENS/LOGO.png}
		
		\vspace{1cm}
		
		{\large \@author\par}
		
		\vfill
		
		{\LARGE\bfseries \@title\par}
		
		\vfill
		
		{\large \@date\par}
		
	\end{titlepage}
}
\makeatother


\begin{document}
	\maketitle
	
	\section{Atividade:}
	Com base no que foi visto na última aula e no PDF ``\texttt{aula-1}'', elabore um documento original em PDF contendo os seguintes elementos:

	
	\begin{itemize}
		\item Um título com data e autor(a).
		\item Ao menos uma seção.
		\item Ao menos uma subseção.
		\item Ao menos uma subsubseção.
		\item Uma fórmula matemática.
	\end{itemize}
	
	\subsection{Elementos opcionais:}
	
	\begin{itemize}
		\item Ao menos uma equação numerada.
		\item Ao menos um parágrafo nomeado.
		\item Uma lista de itens não numerada.
		\item Uma lista de itens numerada.
		\item Uma referência à equação criada no texto.
	\end{itemize}
	
	
	\section{Entrega:}
	
	A entrega será feita pelo seguinte endereço de e-mail: \href{mailto:viniqf@proton.me}{viniqf@proton.me}. \\
	
	O PDF deve ser enviado até o dia 21/01/2026, antes da próxima aula. \\
	
	A aula seguinte começará com a correção das entregas e sugestões de melhoria.

	
\end{document}

