\documentclass{article}

\usepackage{geometry}
\usepackage{microtype}
\usepackage{graphicx}
\usepackage{abntex2cite}
\usepackage[brazil]{babel}
\usepackage[T1]{fontenc}
\usepackage{ragged2e}
\usepackage{enumitem}
\usepackage{indentfirst}
\usepackage{inconsolata}
\usepackage{lmodern}

\setlist[itemize]{itemsep=0.4em}


\geometry{
	a4paper,
	left=3cm,
	right=2cm,
	top=3cm,
	bottom=2cm
}



\author{Vinícius Queiroz Fonseca}
\date{14 de Janeiro de 2026}
\title{Documento de Boas Vindas - LaTeX}

\usepackage{setspace}
\onehalfspacing

\usepackage{titlesec}


% Configurações da Capa
\makeatletter
\renewcommand{\maketitle}{
	\begin{titlepage}
		\centering
		
		\includegraphics[width=5cm]{IMAGENS/LOGO.png}
		
		\vspace{1cm}
		
		{\large \@author\par}
		
		\vfill
		
		{\LARGE\bfseries \@title\par}
		
		\vfill
		
		{\large \@date\par}
		
	\end{titlepage}
}
\makeatother
% --- 

% - Ambiente Textual
\begin{document}
	\justifying
	
	\maketitle
	
	\section*{Documento de Referência}
	
	Esse é um documento preparado com a intenção de organizar as atividades da disciplina do curso de férias. Para o nosso primeiro encontro, é essencial que estejam explicados alguns básicos relacionados ao LaTeX:
	
	\begin{itemize}
		\item O que é o LaTeX?
		\begin{itemize}
			\item De modo geral, o LaTeX pode ser entendido como uma linguagem de programação que é capaz de criar documentos compilados diretamente para PDF ou outro formato de arquivo.
		\end{itemize}
		\item Onde eu construo um documento?
		\begin{itemize}
			\item Você pode escolher um compilador local em conjunto com um editor de texto. 
			
			 Ex: Texlive \cite{texlive} + Texstudio \cite{texstudio}
			
			\item Outra opção é um compilador on-line com sua IDE integrada. 
			
			Ex: Overleaf \cite{overleaf} ou Texlyre \cite{texlyre}.
		\end{itemize}
		\item Quais recursos eu posso usar para aprender além do que estiver presente nesse material?
		\begin{itemize}
			\item Um possível recurso é o website LearnLaTeX, onde existem tutoriais do funcionamento básico ao avançado da linguagem LaTeX \cite{learnlatex}.
			\item Existem tutoriais no YouTube que podem explicar o funcionamento básico da linguagem.
		\end{itemize}
	\item Por que utilizar essa linguagem em vez de um processador de textos convencional, como o Word ou o LibreOffice Writer?
	\begin{itemize}
		\item O LaTeX proporciona uma experiência mais consistente do que editores de texto convencionais, pois o processo de formatação é separado do conteúdo textual. 
		
		Isso garante que o mesmo documento possa ser compilado com exatamente o mesmo formato em diferentes ambientes. 
		
		Além disso, o gerenciamento de citações e referências é superior, uma vez que esse processo é automatizado e centralizado em um único arquivo BibTeX.
	\end{itemize}

	\end{itemize}
	

	
    Com esses elementos em mente, vamos dar continuidade para explicar a estrutura geral dos nossos encontros:
	
	\subsection*{Aula 1 - Dia 14/01/2026:}
	
\begin{itemize}
	\item A primeira aula será parcialmente expositiva, com uma apresentação de slides a respeito do histórico da linguagem LaTeX.
	\item Em sequência, será elaborado um documento básico em tempo real, com um processo de tirada de dúvidas ao longo do caminho.
	\item Por fim, será solicitado um documento de maior complexidade como atividade, que deverá ser enviado por e-mail para avaliação e retorno na aula seguinte.
\end{itemize}

	
	
	\subsection*{Aula 2 - Dia 21/01/2026:}
	
\begin{itemize}
	\item A segunda aula irá explorar conceitos mais avançados da linguagem LaTeX.
	\item Serão abordados o funcionamento dos pacotes adicionais e o gerenciamento de referências bibliográficas.
	\item Também será apresentada uma introdução ao desenvolvimento de documentos utilizando a classe \texttt{abntex2} \cite{abntex2}, seguindo a formatação ABNT.
	\item Por fim, será discutido o uso do template da revista SBC \cite{sbc-template} utilizado nos Trabalhos de Conclusão de Curso dos cursos de Computação da instituição.
\end{itemize}

	
	\bibliographystyle{abntex2-alf}
	\bibliography{boas-vindas}
	
\end{document}
% --- 

