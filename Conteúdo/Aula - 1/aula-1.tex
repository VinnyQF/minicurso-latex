\documentclass{article}

\usepackage{geometry}
\usepackage{microtype}
\usepackage{graphicx}

\author{Vinícius Queiroz Fonseca}
\date{14 de Janeiro de 2025}
\title{Documento de Exemplo - \LaTeX}


\begin{document}

\maketitle

\section{Isso é uma seção:}

Para o \LaTeX, seções são numeradas automaticamente. Elas contém um título maior que as subseções e subsubseções. \\

Quem determina a ordem de numeração é o chamado do comando ``$\backslash$section \{\}'', ou seja, qualquer seção que for criada após uma seção original será numerada em ordem crescente.

\section{Exemplo de uma nova seção:}

Como pode ser observado, essa seção já apresenta a numeração correta.

\subsection{Isso é uma subseção:}

Subseções seguem o mesmo princípio de funcionamento, iterando em 0.1 em sequência à última seção conhecida. Elas podem ser chamadas pelo comando ``$\backslash$subsection\{\}''.

\subsubsection{Isso é uma subsubseção:}

Por fim. Nós temos as subsubseções, que são iteradas em 0.0.1, em sequência à uma última seção ou subseção conhecida. Normalmente subsubseções são precedidas de uma seção, mas isso não necessariamente é algo obrigatório. Elas podem ser chamadas pelo comando ``$\backslash$subsection\{\}''

\paragraph{Isso é um parágrafo:}

\begin{itemize}
	\item Esses são items pontuados
	\item Podem existir quantos pontos forem necessários
\end{itemize}


\end{document}