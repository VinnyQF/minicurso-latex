\documentclass{article}

\usepackage{geometry}
\usepackage{microtype}
\usepackage{graphicx}
\usepackage[brazil]{babel}
\usepackage[T1]{fontenc}
\usepackage{lmodern}
\usepackage{listings}
\usepackage{setspace}
\usepackage[colorlinks=false, linkcolor=blue]{hyperref}
\usepackage{fancyhdr}
\setlength{\parindent}{0pt}
\pagestyle{fancy} 

% Limpa os campos padrão
\fancyhf{}



% Rodapé
\fancyfoot[L]{Centro Universitário Jorge Amado}
\fancyfoot[C]{\thepage}
\fancyfoot[R]{Vinícius Queiroz Fonseca}

\renewcommand{\headrulewidth}{0.4pt}
\renewcommand{\footrulewidth}{0.4pt}  



\lstset{
	basicstyle=\ttfamily\small,
	frame=single,
	breaklines=true,
	showstringspaces=false,
	extendedchars=true,
	inputencoding=utf8,
	literate=
	{á}{{\'a}}1
	{à}{{\`a}}1
	{ã}{{\~a}}1
	{â}{{\^a}}1
	{é}{{\'e}}1
	{ê}{{\^e}}1
	{í}{{\'i}}1
	{ó}{{\'o}}1
	{ô}{{\^o}}1
	{õ}{{\~o}}1
	{ú}{{\'u}}1
	{ç}{{\c c}}1
	{Á}{{\'A}}1
	{À}{{\`A}}1
	{Ã}{{\~A}}1
	{Â}{{\^A}}1
	{É}{{\'E}}1
	{Ê}{{\^E}}1
	{Í}{{\'I}}1
	{Ó}{{\'O}}1
	{Ô}{{\^O}}1
	{Õ}{{\~O}}1
	{Ú}{{\'U}}1
	{Ç}{{\c C}}1
}

\geometry{
	a4paper,
	left=3cm,
	right=2cm,
	top=3cm,
	bottom=2cm
}


% Configurações da Capa
\author{Vinícius Queiroz Fonseca}
\date{14 de Janeiro de 2026}
\title{Aula 1 - Tutorial de \LaTeX \hspace{1pt} básico}

\makeatletter
\renewcommand{\maketitle}{
	\begin{titlepage}
		\centering
		
		\includegraphics[width=5cm]{IMAGENS/LOGO.png}
		
		\vspace{1cm}
		
		{\large \@author\par}
		
		\vfill
		
		{\LARGE\bfseries \@title\par}
		
		\vfill
		
		{\large \@date\par}
		
	\end{titlepage}
}
\makeatother


\begin{document}

\maketitle

\tableofcontents

\newpage

\section{Introdução:}
Esse documento serve como manual de consulta para a primeira aula do minicurso e pode ser consultado a qualquer momento como referência para criar um documento de exemplo inicial.


\section{A Sintaxe da Linguagem \TeX:}

O LaTeX é uma linguagem baseada em \textbf{comandos} e \textbf{ambientes}.  \\

Tudo o que você escreve fora de um comando é considerado texto normal, enquanto os comandos instruem o compilador a realizar formatações, criar estruturas ou inserir elementos especiais.

\subsection{Comandos ou Macros:}

Um comando, também conhecido como ``macro'' em LaTeX geralmente começa com uma \textbf{barra invertida} ``\textbackslash'' seguida do nome do comando, podendo ou não receber \textbf{argumentos} entre chaves \{\}.  \\

Essa é a sintaxe básica de um comando:

\begin{lstlisting}
	\comando
\end{lstlisting}

Caso esse comando tenha opções, essa é a sintaxe:

\begin{lstlisting}
	\comando[parâmetro opcional]{parâmetro obrigatório}
\end{lstlisting}

\subsection{Ambientes:}

Ambientes delimitam blocos de conteúdo e são definidos entre os comandos \texttt{\textbackslash begin\{\}} e \texttt{\textbackslash end\{\}}. \\

Essa é a sintaxe básica de um ambiente genérico:

\begin{lstlisting}
	\begin{ambiente}      % Início do ambiente
	conteúdo...
	\end{ambiente}         % Fim do ambiente
\end{lstlisting}

\subsection{Comentários:}

Comentários em LaTeX são iniciados com o símbolo ``\%''. Tudo que estiver na linha após este símbolo é ignorado pelo compilador.

\begin{lstlisting}
	% Isto é um comentário e não aparece no PDF
\end{lstlisting}

\textbf{Em resumo:} A estrutura básica do LaTeX envolve:

\begin{itemize}
	\item Texto normal: conteúdo que aparecerá no documento sem formatação especial.
	\item Comandos: instruções que alteram a formatação ou criam elementos.
	\item Ambientes: blocos de conteúdo delimitados, cada um com funções específicas.
	\item Comentários: texto ignorado pelo compilador para documentar o código.
\end{itemize}

\section{Criando um Preâmbulo do Documento:}

Antes de criar um documento em si, é importante entender o conceito por trás de um preâmbulo. De modo geral, o preâmbulo é onde ficam todas as configurações associadas ao texto e à sua formatação, mas que não participam diretamente do documento. \\

Fazendo uma analogia, é como importar bibliotecas e configurações em uma linguagem de programação tradicional, e existem vários pacotes que podem ser importados durante o preâmbulo. Para os propósitos desta aula, não iremos abordar diretamente quais pacotes importar, ou como esse processo funciona, e sim alguns essenciais para entendimento básico da criação de um documento. \\

Quais itens fazem parte de um preâmbulo?

\begin{itemize}
	\item Classe do documento (artigo, relatório, livro, etc.)
	\item Idioma
	\item Fonte
	\item Margens
	\item Pacotes adicionais
\end{itemize}

O preâmbulo fica localizado \textbf{antes} do comando \texttt{\textbackslash begin\{document\}}. Ou seja, antes da declaração do ambiente de escrita em si.

\subsection{Classe do Documento:}

Um documento pode ter diversas possibilidades de classe, e apenas uma classe por vez. Um exemplo disso é a classe que será usada nesta aula, conhecida como artigo, ou \texttt{article}. \\

A classe \texttt{article} é indicada para documentos curtos, artigos, trabalhos acadêmicos simples e relatórios pequenos.

Outras classes comuns incluem:
\begin{itemize}
	\item \texttt{report}
	\item \texttt{book}
	\item \texttt{beamer}
\end{itemize}

Para escolher uma classe de um documento, basta adicionar uma linha no preâmbulo contendo a informação geral dessa classe. Por exemplo:

\begin{lstlisting}
	\documentclass{article}
\end{lstlisting}

\subsection{Importando Pacotes:}

Importar pacotes é um passo fundamental, porém nebuloso em alguns aspectos. Existem muitos pacotes no LaTeX, muitos desses realizam funções excelentes e permitem alta capacidade de customização. \\
 
O porém mora em uma ausência de documentação direta, então a melhor forma de aprender a utilizar um pacote corretamente é buscando por recursos na internet que te expliquem como aplicar uma determinada funcionalidade, e, depois disso, memorizar este conteúdo. Essa é sintaxe de importação de um pacote:



\begin{lstlisting}
	\usepackage{pacote}
\end{lstlisting}

Alguns pacotes podem conter opções adicionais. Nesses casos, podem haver situações como essa:

\begin{lstlisting}
	\usepackage[opções]{pacote}
\end{lstlisting}

Para a construção do seu primeiro documento, esta lista de pacotes deve ser o suficiente:


\begin{lstlisting}
	\usepackage[brazil]{babel} % Suporte ao idioma português
	\usepackage[T1]{fontenc} % Codificação adequada da fonte fonte
	\usepackage{lmodern}  % Fonte Latin Modern
	\usepackage{graphicx} % Inserção de imagens
	\usepackage{microtype} % Melhora a tipografia do texto
\end{lstlisting}

É importante ressaltar que um preâmbulo pode crescer a depender da quantidade de pacotes que um documento utilizar. Uma forma de mitigar o tamanho do preâmbulo é fazendo importações múltiplas de pacotes, utilizando a seguinte sintaxe:

\begin{lstlisting}
	\usepackage[opção global]{pacote1, pacote2, pacote...}
\end{lstlisting}

Embora seja possível importar vários pacotes em uma única linha, as opções informadas são aplicadas a todos os pacotes listados. Por isso, essa forma deve ser usada com cautela, especialmente quando os pacotes exigem configurações distintas.

\section{Criando o Ambiente de Documento:}

O primeiro passo de construção de um documento simples em LaTeX é a criação do ambiente \texttt{document}. \\

Para fazer isso, basta adicionar uma estrutura semelhante a essa como o primeiro elemento do seu documento:

\begin{lstlisting}
	\begin{document}
		Todo o resto do documento aqui.
	\end{document}
\end{lstlisting}


\section{Criando uma Seção:}

Para o LaTeX, seções são numeradas automaticamente. Elas contém um título maior que as subseções e subsubseções. \\

Quem determina a ordem de numeração é o chamado do comando \texttt{\textbackslash section\{\}}, ou seja, qualquer seção que for criada após uma seção original será numerada em ordem crescente, seguindo o padrão \{1\}.

\begin{lstlisting}
	\section{Exemplo de Seção}
\end{lstlisting}

\section{Exemplo de uma nova seção:}

Como pode ser observado, essa seção já apresenta a numeração correta.

\subsection{Criando uma Subseção:}

Subseções seguem o mesmo princípio de funcionamento, iterando em \{1.1\} em sequência à última seção conhecida. Elas podem ser chamadas pelo comando \texttt{\textbackslash subsection\{\}}.

\begin{lstlisting}
	\subsection{Exemplo de Subseção}
\end{lstlisting}

\subsubsection{Criando uma Subsubseção:}

Por fim. Nós temos as subsubseções, que são iteradas em \{1.1.1\}, em sequência à uma última seção ou subseção conhecida. Normalmente subsubseções são precedidas de uma seção, mas isso não necessariamente é algo obrigatório. Elas podem ser chamadas pelo comando \texttt{\textbackslash subsubsection\{\}}.

\begin{lstlisting}
	\subsubsection{Exemplo de Subsubseção}
\end{lstlisting}

\paragraph{Isso é um Parágrafo:}

O comando \texttt{\textbackslash paragraph\{\}} cria um título de parágrafo. Seu comportamento visual pode variar conforme a classe do documento. A quebra de linha não acontece por padrão quando um título de parágrafo é inserido.
\begin{lstlisting}
	\paragraph{Exemplo de Parágrafo com título}
\end{lstlisting}

\section{Listas:}

\subsection{Listas Não Ordenadas:}
O ambiente \texttt{itemize} é utilizado para criar listas não ordenadas, conforme o exemplo abaixo:

\begin{lstlisting}
	\begin{itemize}
		\item Primeiro itemLaTeX
		\item Segundo item
		\item Terceiro item
	\end{itemize}
\end{lstlisting}


\begin{itemize}
	\item Este é um item da lista
	\item É possível adicionar quantos itens forem necessários
	\item O alinhamento e espaçamento são gerenciados automaticamente
\end{itemize}



\subsection{Listas Ordenadas:}
O ambiente \texttt{enumerate} é utilizado para criar listas ordenadas, nas quais os itens são numerados automaticamente, conforme o exemplo abaixo:

\begin{lstlisting}
	\begin{enumerate}
		\item Primeiro item
		\item Segundo item
		\item Terceiro item
	\end{enumerate}
\end{lstlisting}

\begin{enumerate}
	\item Este é o primeiro item da lista
	\item A numeração é gerada automaticamente
	\item A ordem dos itens é preservada
\end{enumerate}

\section{Estilização do Texto:}

Quando você escreve seu documento, irão surgir necessidades como a definição de caracteres em itálico, negrito, texto técnico e outras possibilidades. \\

Cada uma dessas possibilidades conta com uma determinada macro que consegue encapsular o texto e formatá-lo da forma desejada. \\

Agora, vamos elaborar sobre as formatações mais usadas:

\subsection{Texto em Negrito:}

Para tornar o texto em \textbf{negrito}, basta encapsular uma palavra com a macro \texttt{$\backslash$textbf\{\}.}

\begin{lstlisting}
	\textbf{texto em negrito.}
\end{lstlisting}

\subsection{Texto em Itálico:}

Para tornar o texto em \textit{itálico}, basta encapsular uma palavra com a macro \texttt{$\backslash$textit\{\}.}

\begin{lstlisting}
	\textit{texto em itálico.}
\end{lstlisting}

\subsection{Texto Técnico:}

Existe também uma opção para formatar o texto em um formato \texttt{técnico}, com o objetivo de diferenciar a informação original, sem usar negrito ou itálico. \\

Isso é feito através da macro \texttt{$\backslash$texttt\{\}}

\begin{lstlisting}
	\texttt{texto técnico.}
\end{lstlisting}

\subsubsection{Estilização Múltipla:}

Caso seja necessário, é possível misturar esses elementos. Por exemplo, \textbf{\textit{é possível produzir um texto itálico em negrito.}} Para fazer isso, basta combinar as macros, encapsulando uma dentro da outra. \\

Ou seja, fazer algo conforme o exemplo abaixo:

\begin{lstlisting}
	\textbf{\textit{texto em negrito e em itálico.}}
\end{lstlisting}

A ordem em si em que esses elementos aparecem não importa, \textit{\textbf{então é possível fazer o contrário e obter o mesmo efeito final:}}

\begin{lstlisting}
	\textit{\textbf{texto em itálico e depois em negrito}}
\end{lstlisting}


\section{O Ambiente Matemático:}

O LaTeX possui um ambiente próprio para a escrita de expressões matemáticas, o que permite criar fórmulas com excelente qualidade tipográfica, de forma consistente e padronizada. \\

Existem duas formas principais de escrever expressões matemáticas: em linha (inline) e em destaque (display).

\subsection{Modo Matemático em Linha:}

O modo matemático em linha é utilizado quando a expressão faz parte do texto corrido. Ele é delimitado pelo símbolo \texttt{\$}.

\begin{lstlisting}
	A área de um quadrado é dada por $A = l^2$.
\end{lstlisting}

O resultado será uma expressão matemática integrada ao texto, sem quebra de linha. Ex: $A = l^2$

\subsection{Modo Matemático em Destaque:}

Quando a expressão precisa de mais destaque ou ocupa mais espaço, utiliza-se o modo matemático em destaque, delimitado por \texttt{\textbackslash[} e \texttt{\textbackslash]}.

\[
A = l^2
\]

\begin{lstlisting}
	\[
	A = l^2
	\]
\end{lstlisting}


\subsection{Modo de Equação Numerada:}

O LaTeX permite criar equações numeradas automaticamente usando o ambiente \texttt{equation}. Isso é útil quando você deseja referenciar a equação posteriormente no texto.

\begin{equation}
	E = mc^2
	\label{exemplo_1}
\end{equation}

\begin{lstlisting}
	\begin{equation}
		E = mc^2
		\label{exemplo_1}		
	\end{equation}
\end{lstlisting}


\subsubsection{Referenciando Elementos Numerados:}

Para criar referências automáticas, usamos: \\

\texttt{\textbackslash label{}} dentro do ambiente numerado (equação, figura, tabela, etc.). \\

\texttt{\textbackslash ref{}} fora do ambiente, onde queremos mencionar o número. \\

Exemplo com equação:

\begin{lstlisting}
	\begin{equation}
		E = mc^2
		\label{eq:energia}
	\end{equation}
	
	Como mostrado na equação~\ref{eq:energia}, ...
\end{lstlisting}

Exemplo com figura:

\begin{lstlisting}
	\begin{figure}[h]
		\centering
		\includegraphics[width=0.5\textwidth]{exemplo.png}
		\caption{Exemplo de figura}
		\label{fig:exemplo}
	\end{figure}
	
	A figura~\ref{fig:exemplo} ilustra o conceito.
\end{lstlisting}

Dessa forma, os números de equações, figuras e tabelas são atualizados automaticamente caso novos elementos sejam adicionados. Isso evita que você precise renumerar manualmente o documento.


\subsection{Expoentes e Índices:}

Expoentes são criados com o caractere \texttt{\^} e índices com o caractere \texttt{\_}.


\[
x^2 + y_1
\]


\begin{lstlisting}
	\[
	x^2 + y_1
	\]
\end{lstlisting}


Quando o expoente ou índice possui mais de um caractere, é necessário utilizar chaves.

\[
x^{n+1} + y_{i,j}
\]

\begin{lstlisting}
	\[
	x^{n+1} + y_{i,j}
	\]
\end{lstlisting}



\subsection{Frações:}

Frações são criadas com o comando \texttt{\textbackslash frac\{\}\{\}}.

\[
\frac{a}{b}
\]

\begin{lstlisting}
	\[
	\frac{a}{b}
	\]
\end{lstlisting}

Também é possível combinar frações com outras operações matemáticas.

\[
\frac{x^2 + 1}{x - 1}
\]

\begin{lstlisting}
	\[
	\frac{x^2 + 1}{x - 1}
	\]
\end{lstlisting}

\subsection{Raízes:}

A macro \texttt{\textbackslash sqrt\{\}} é utilizada para criar raízes quadradas.

\[
\sqrt{x}
\]

\begin{lstlisting}
	\[
	\sqrt{x}
	\]
\end{lstlisting}

Para raízes de ordem diferente de dois, utiliza-se um argumento opcional.

\[
\sqrt[3]{x}
\]

\begin{lstlisting}
	\[
	\sqrt[3]{x}
	\]
\end{lstlisting}

\subsection{Símbolos Matemáticos Básicos:}

O \LaTeX{} fornece diversos símbolos matemáticos prontos para uso.

\[
\alpha + \beta = \gamma
\]

\begin{lstlisting}
	\[
	\alpha + \beta = \gamma
	\]
\end{lstlisting}

\[
a \leq b \quad c \geq d \quad x \neq y
\]

\begin{lstlisting}
	\[
	a \leq b \quad c \geq d \quad x \neq y
	\]
\end{lstlisting}

\subsection{Observação Importante:}

O modo matemático ignora espaços e utiliza regras próprias de formatação. Por isso, comandos matemáticos devem sempre estar dentro de um ambiente matemático. \\

Tentativas de usar comandos como \texttt{\textbackslash frac} ou \texttt{\textbackslash sqrt} fora do modo matemático resultarão em erro de compilação.

\subsubsection{Extra: Lista de Símbolos e Comandos Mais Usados}

Segue aqui uma lista de elementos muito usados no ambiente matemático do LaTeX. É bom notar que existem muitos elementos que vão além desse ponto, e que é importante consultar a documentação para elementos mais avançados.

\paragraph{Operações Básicas:}

\begin{itemize}
	\item Soma: \texttt{+}
	\item Subtração: \texttt{-}
	\item Multiplicação: \texttt{\textbackslash times} \quad ($\times$)
	\item Divisão: \texttt{\textbackslash div} \quad ($\div$)
\end{itemize}

\paragraph{Relações:}

\begin{itemize}
	\item Igualdade: \texttt{=} \quad ($=$)
	\item Diferente: \texttt{\textbackslash neq} \quad ($\neq$)
	\item Menor ou igual: \texttt{\textbackslash leq} \quad ($\leq$)
	\item Maior ou igual: \texttt{\textbackslash geq} \quad ($\geq$)
\end{itemize}

\paragraph{Expoentes, Índices e Frações:}

\begin{itemize}
	\item Expoente: \texttt{x\textasciicircum 2} \quad ($x^2$)
	\item Índice: \texttt{x\_1} \quad ($x_1$)
	\item Fração: \texttt{\textbackslash frac\{a\}\{b\}} \quad ($\frac{a}{b}$)
\end{itemize}

\paragraph{Raízes:}

\begin{itemize}
	\item Raiz quadrada: \texttt{\textbackslash sqrt\{x\}} \quad ($\sqrt{x}$)
	\item Raiz cúbica: \texttt{\textbackslash sqrt[3]\{x\}} \quad ($\sqrt[3]{x}$)
\end{itemize}

\paragraph{Letras Gregas:}

\begin{itemize}
	\item \texttt{\textbackslash alpha} \quad ($\alpha$)
	\item \texttt{\textbackslash beta} \quad ($\beta$)
	\item \texttt{\textbackslash gamma} \quad ($\gamma$)
	\item \texttt{\textbackslash pi} \quad ($\pi$)
\end{itemize}

\paragraph{Espaçamento no Modo Matemático:}

\begin{itemize}
	\item Pequeno espaço: \texttt{\textbackslash ,}
	\item Espaço médio: \texttt{\textbackslash ;}
	\item Espaço grande: \texttt{\textbackslash quad}
\end{itemize}


Todos os comandos apresentados nesta seção devem ser utilizados dentro do modo matemático, seja ele em linha (\texttt{\$...\$}) ou em destaque (\texttt{\textbackslash[...\textbackslash]}).



\end{document}